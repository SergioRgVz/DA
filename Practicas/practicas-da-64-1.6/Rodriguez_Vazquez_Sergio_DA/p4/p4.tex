\documentclass[]{article}

\usepackage[left=2.00cm, right=2.00cm, top=2.00cm, bottom=2.00cm]{geometry}
\usepackage[spanish,es-noshorthands]{babel}
\usepackage[utf8]{inputenc} % para tildes y ñ

%opening
\title{Práctica 4. Exploración de grafos}
\author{Sergio Rpdríguez Vázquez \\ % mantenga las dos barras al final de la línea y este comentario
sergio.rodriguezvazquez@alum.uca.es \\ % mantenga las dos barras al final de la línea y este comentario
Teléfono: +634471641 \\ % mantenga las dos barras al final de la linea y este comentario
NIF: 77171745Y \\ % mantenga las dos barras al final de la línea y este comentario
}


\begin{document}

\maketitle

%\begin{abstract}
%\end{abstract}

% Ejemplo de ecuación a trozos
%
%$f(i,j)=\left\{ 
%  \begin{array}{lcr}
%      i + j & si & i < j \\ % caso 1
%      i + 7 & si & i = 1 \\ % caso 2
%      2 & si & i \geq j     % caso 3
%  \end{array}
%\right.$

\begin{enumerate}
\item Comente el funcionamiento del algoritmo y describa las estructuras necesarias para llevar a cabo su implementación.

Utilizo una estructura llamada C, que guarda la posición y el valor dado por defaultCellValue.

En el primer caso, el algoritmo sin preordenación maneja una lista de C y va buscando la mejor posición en cada caso, comprobando que sea factible hasta que o bien termina la lista de C o ya no hay mas defensas que colocar.

En los demás casos, los algoritmos manejan un vector de C.

\item Incluya a continuación el código fuente relevante del algoritmo.

Utilizo primero una estructura creada por mí que me relaciona una defensa con el coste asignado previamente por mi estrategia. Esta estructura se llama DefenseCost, al comenzar el problema, recorro la lista de defensas y guardo en un vector de la STL todos los objetos de tipo DefenseCost para tener así todos los correspondientes costes.

\begin{lstlisting}
//Estructura DefenseCost
struct DefenseCost
{
    DefenseCost();
    DefenseCost(Defense *defense, float cost): defense_(defense), cost_(cost){}
    Defense *defense_;
    float cost_;
};

    //Bucle para rellenar los "candidatos" del problema
    //Quitamos la primera defensa porque no la tenemos que tener en cuenta
 
    selectedIDs.push_back((*it)->id);
    ases = ases - (*it)->cost;
    it++;


    //Recorremos la lista de defensas que nos dan y vamos creando la lista de candidatos para aplicar el algoritmo de la mochila
    while(it != defenses.end())
    {
        DefenseCost defenseiterated((*it), calculatecost((*it)));
        defensecosts.push_back(defenseiterated);
        ++it;
    }



\end{lstlisting}


\end{enumerate}

Todo el material incluido en esta memoria y en los ficheros asociados es de mi autoría o ha sido facilitado por los profesores de la asignatura. Haciendo entrega de esta práctica confirmo que he leído la normativa de la asignatura, incluido el punto que respecta al uso de material no original.

\end{document}
