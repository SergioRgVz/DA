\documentclass[]{article}

\usepackage[left=2.00cm, right=2.00cm, top=2.00cm, bottom=2.00cm]{geometry}
\usepackage[spanish,es-noshorthands]{babel}
\usepackage[utf8]{inputenc} % para tildes y ñ
\usepackage{graphicx} % para las figuras
\usepackage{xcolor}
\usepackage{listings} % para el código fuente en c++

\lstdefinestyle{customc}{
  belowcaptionskip=1\baselineskip,
  breaklines=true,
  frame=single,
  xleftmargin=\parindent,
  language=C++,
  showstringspaces=false,
  basicstyle=\footnotesize\ttfamily,
  keywordstyle=\bfseries\color{green!40!black},
  commentstyle=\itshape\color{gray!40!gray},
  identifierstyle=\color{black},
  stringstyle=\color{orange},
}
\lstset{style=customc}


%opening
\title{Práctica 2. Programación dinámica}
\author{Sergio Rpdríguez Vázquez \\ % mantenga las dos barras al final de la línea y este comentario
sergio.rodriguezvazquez@alum.uca.es \\ % mantenga las dos barras al final de la línea y este comentario
Teléfono: +634471641 \\ % mantenga las dos barras al final de la linea y este comentario
NIF: 77171745Y \\ % mantenga las dos barras al final de la línea y este comentario
}


\begin{document}

\maketitle

%\begin{abstract}
%\end{abstract}

% Ejemplo de ecuación a trozos
%
%$f(i,j)=\left\{ 
%  \begin{array}{lcr}
%      i + j & si & i < j \\ % caso 1
%      i + 7 & si & i = 1 \\ % caso 2
%      2 & si & i \geq j     % caso 3
%  \end{array}
%\right.$

\begin{enumerate}
\item Formalice a continuación y describa la función que asigna un determinado valor a cada uno de los tipos de defensas.

Utilizo una estructura llamada C, que guarda la posición y el valor dado por defaultCellValue.

En el primer caso, el algoritmo sin preordenación maneja una lista de C y va buscando la mejor posición en cada caso, comprobando que sea factible hasta que o bien termina la lista de C o ya no hay mas defensas que colocar.

En los demás casos, los algoritmos manejan un vector de C.

\item Describa la estructura o estructuras necesarias para representar la tabla de subproblemas resueltos.

Utilizo primero una estructura creada por mí que me relaciona una defensa con el coste asignado previamente por mi estrategia. Esta estructura se llama DefenseCost, al comenzar el problema, recorro la lista de defensas y guardo en un vector de la STL todos los objetos de tipo DefenseCost para tener así todos los correspondientes costes.

\begin{lstlisting}
//Estructura DefenseCost
struct DefenseCost
{
    DefenseCost();
    DefenseCost(Defense *defense, float cost): defense_(defense), cost_(cost){}
    Defense *defense_;
    float cost_;
};

    //Bucle para rellenar los "candidatos" del problema
    //Quitamos la primera defensa porque no la tenemos que tener en cuenta
 
    selectedIDs.push_back((*it)->id);
    ases = ases - (*it)->cost;
    it++;


    //Recorremos la lista de defensas que nos dan y vamos creando la lista de candidatos para aplicar el algoritmo de la mochila
    while(it != defenses.end())
    {
        DefenseCost defenseiterated((*it), calculatecost((*it)));
        defensecosts.push_back(defenseiterated);
        ++it;
    }



\end{lstlisting}

\item En base a los dos ejercicios anteriores, diseñe un algoritmo que determine el máximo beneficio posible a obtener dada una combinación de defensas y \emph{ases} disponibles. Muestre a continuación el código relevante.

Escriba aquí su respuesta al ejercicio 3.

\item Diseñe un algoritmo que recupere la combinación óptima de defensas a partir del contenido de la tabla de subproblemas resueltos. Muestre a continuación el código relevante.

Las características del algoritmo voraz en este caso son:

\begin{itemize}
    \item Candidatos: representados por una estructura que guarda un Vector3 position de la celda y el valor de la celda (la puntuación).
    \item Candidatos seleccionados: la lista de candidatos se ordena y seleccionamos los candidatos con la mejor puntuación, en nuestro caso la mejor puntuación para la primera defensa sería la mayor distancia a las esquinas, y la mejor posición para las demás defensas sería la menor distancia a la defensa extractora.
    \item Solución: en nuestro caso, no tenemos función solución explícita sino que tenemos que comprobar que todas las defensas que tenemos se hayan colocado, en ese caso habríamos llegado a la solución.
    \item Selección: en nuestro caso la función de selección sería escoger al mejor candidato de la lista ya ordenada
    \item Función factibilidad: tenemos la función factible que comprueba si una posición es válida o no para una defensa
    \item Función objetivo: la función objetivo sería la función de cellValue, que puntúa las distintas posiciones según la distancia a las esquinas, o a la defensa extractora
    \item Objetivo del algoritmo: colocar en las defensas en las mejores posiciones para que aguante el mayor tiempo posible en los distintos niveles


\end{itemize}

\end{enumerate}

Todo el material incluido en esta memoria y en los ficheros asociados es de mi autoría o ha sido facilitado por los profesores de la asignatura. Haciendo entrega de este documento confirmo que he leído la normativa de la asignatura, incluido el punto que respecta al uso de material no original.

\end{document}
